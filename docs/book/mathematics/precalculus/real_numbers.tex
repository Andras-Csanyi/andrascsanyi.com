\section{Real numbers}

\subsection{Properties of Real Numbers}

\begin{tabularx}{1\textwidth}{
    p{\dimexpr0.5\textwidth\relax}
    p{\dimexpr0.5\textwidth\relax}
}
\toprule
\multicolumn{2}{c}{\textbf{Properties of real numbers}} \\
\midrule

\textbf{Property} & \textbf{Example}\\
\midrule

\makecell[l]{
    \textbf{Commutative Properties} \\ 1[ex]
    $a + b = b + a$ \\ 
    $a \cdot b = b \cdot a$
} 
& 
\makecell[l]{
    $4 + 3 = 3 + 4$ \\ 
    $4 \cdot 3 = 3 \cdot 4$
} 
\\
\midrule

\makecell[l]{
    \textbf{Associative Properties} \\[1ex] 
    $(a + b) + c = a + (b + c)$ \\ 
    $(a \cdot b) \cdot c = a \cdot (b \cdot c)$
} 
&
\makecell[l]{
    $(2 + 3) + 4 = 2 + (3 + 4)$ \\ 
    $(2 \cdot 3) \cdot 4 = (2 \cdot 3) \cdot 4$
} 
\\

\midrule

\makecell[l]{
    \textbf{Distributive Properties} \\[1ex] 
    $a(b + c) = a \cdot b + a \cdot c$ \\ 
    $(b + c) \cdot a = a \cdot b + a \cdot c$
} 
& 
\makecell[l]{
    $2(3 + 4) = 2 \cdot 3 + 2 \cdot 4$ \\ 
    $(3 + 4) \cdot 2 = 3 \cdot 2 + 4 \cdot 2$
} 
\\  

\bottomrule
\end{tabularx}

\subsection{Addition and Subtraction}

\begin{tabularx}{1\textwidth}{
    p{\dimexpr0.5\textwidth\relax}
    p{\dimexpr0.5\textwidth\relax}
}
\toprule
\multicolumn{2}{c}{\textbf{Rules of Addition and Subtraction}} \\
\midrule

\textbf{Property} & \textbf{Example}\\
\midrule

\makecell[l]{
    $(-1)a = -a$
} 
& 
\makecell[l]{
    $(-1)5 = -5$
} 
\\
\makecell[l]{
    $-(-a) = a$
} 
& 
\makecell[l]{
    $-(-5) = 5$
} 
\\
\makecell[l]{
    $(-a)b = a(-b) = -(ab)$
} 
& 
\makecell[l]{
    $(-3)5 = 5(-3) = -(5 \cdot 3)$
} 
\\
\makecell[l]{
    $(-a)(-b) = ab$
} 
& 
\makecell[l]{
    $(-3)(-5) = 5 \cdot 3$
} 
\\
\makecell[l]{
    $-(a+b) = -a-b$
} 
& 
\makecell[l]{
    $-(3+5) = -5-3$
} 
\\
\makecell[l]{
    $-(a-b) = b-a = -a + b$
} 
& 
\makecell[l]{
    $-(3-5) = 5-3 = -3+5$
} 
\\

\bottomrule
\end{tabularx}

\subsection{Multiplication and Division}

\begin{tabularx}{1\textwidth}{
    p{\dimexpr0.5\textwidth\relax}
    p{\dimexpr0.5\textwidth\relax}
}
\toprule
\multicolumn{2}{c}{\textbf{Rules of Multiplication and Division}} \\
\midrule

\textbf{Property} & \textbf{Example}\\
\midrule

\makecell[l]{
    \vspace{5pt}
    1, $ \frac{a}{b} \cdot \frac{c}{d} = \frac{ac}{bd} $
    \vspace{5pt}
} 
& 
\makecell[l]{
    \vspace{5pt}
    $ \frac{3}{2} \cdot \frac{5}{4} = \frac{3 \cdot 5}{2 \cdot 4} = \frac{15}{8}$
    \vspace{5pt}
} 
\\
\makecell[l]{
    \vspace{5pt}
   2, $ \frac{a}{b} \div \frac{c}{d} = \frac{a}{b} \cdot \frac{d}{c} $
    \vspace{5pt}
} 
& 
\makecell[l]{
    \vspace{5pt}
    $ \frac{3}{2} \div \frac{5}{4} = \frac{3}{2} \cdot \frac{4}{5} = \frac{12}{10}$
    \vspace{5pt}
} 
\\
\makecell[l]{
    \vspace{5pt}
   3, $ \frac{a}{c} + \frac{b}{c} = \frac{a + b}{c} $
    \vspace{5pt}
} 
& 
\makecell[l]{
    \vspace{5pt}
    $ \frac{3}{2} + \frac{5}{2} = \frac{3 + 5}{2}$
    \vspace{5pt}
} 
\\
\makecell[l]{
    \vspace{5pt}
   4, $ \frac{a}{b} + \frac{c}{d} = \frac{ad + cb}{bd} $
    \vspace{5pt}
} 
& 
\makecell[l]{
    \vspace{5pt}
    $ \frac{3}{2} + \frac{5}{3} = \frac{ \left( 3 \cdot 3 \right) + \left( 2 \cdot 5 \right) }{ 2 \cdot 3 }$
    \vspace{5pt}
} 
\\
\makecell[l]{
    \vspace{5pt}
    5,  $ \frac{ac}{bc} = \frac{a}{b} $
    \vspace{5pt}
} 
& 
\makecell[l]{
    \vspace{5pt}
    $ \frac{3 \cdot 4}{2 \cdot 4} = \frac{3}{2}$
    \vspace{5pt}
} 
\\
\makecell[l]{
    \vspace{5pt}
   6, $ \text{If } \frac{a}{b} = \frac{c}{d}, \text{then } ad = bc$
    \vspace{5pt}
} 
& 
\makecell[l]{
    \vspace{5pt}
    $ \frac{2}{4} = \frac{4}{8}, \text{so } 2 \cdot 8 = 4 \cdot 4$
    \vspace{5pt}
} 
\\
\end{tabularx}

\subsection{The Real Line}

The properties of the Real Line:

\begin{itemize}
    \item reference point is called origin, its value is $0$
    \item each positive $x$ number is represented on the right side
    \item each negative $-x$ number is represented on the left side
    \item the number assiciated with the point $P$ is called the coordinate of
        $P$
    \item the real numbers are ordered:
    \begin{itemize}
        \item $a$ is less than $b$,  $a < b$ if $b - a$ is a positive number,
            meaning $a$ lies on the left of $b$ on the number line
        \item $a$ is greater than $b$, $a > b$ when $a - b$ is a negative
            number, meaning $a$ lies on the right side of $b$ on the number line
    \end{itemize}
\end{itemize}

\subsection{Sets and Intervals}

A \textbf{Set} is a collection of objects called \textbf{elements}.

Notation:
\begin{itemize}
    \item $S$ is a set
    \item $a \in S$ means $a$ is element of $S$
    \item $b \notin S$ means $b$ is not element of $S$
    \item $S$ and $T$ are sets and their union is $S \cup T$ and consists all of
        elements that either member of $S$ or $T$ or both
    \item the intersection of $S \cap T$ means the elements part of both sets, in
        other words, this is the common part of both sets
    \item the empty set is denoted by $\emptyset$
\end{itemize}

The listing the elements notation of a set:

\begin{center}
$A = \{1, 2, 3, 4, 5\}$
\end{center}

The builder notion of a set: 

\begin{center}
$ A = \{x | x \text{ is an integer and } 0 < x < 8 \}$
\end{center}
